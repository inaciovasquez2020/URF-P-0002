\documentclass[11pt]{article}

\usepackage[T1]{fontenc}
\usepackage{lmodern}
\usepackage{amsmath,amssymb,amsthm}
\usepackage{geometry}
\usepackage{hyperref}
\geometry{margin=1in}

% --- Theorem Environments (MANDATORY) ---
\theoremstyle{definition}
\newtheorem{definition}{Definition}[section]

\theoremstyle{plain}
\newtheorem{lemma}[definition]{Lemma}
\newtheorem{theorem}[definition]{Theorem}
\newtheorem{corollary}[definition]{Corollary}

\title{Structural Capacity Limits Beyond Local Refinement}
\author{Inacio F. Vasquez \\ Independent Researcher}
\date{}

\begin{document}
\maketitle

\hrule
\vspace{0.5em}
\noindent
\textbf{STATUS:} DRAFT / IN PROGRESS \\
\textbf{SCOPE:} Structural limits of adaptive and multi-context refinement systems under bounded information capacity. \\
\textbf{DEPENDENCIES:} Shannon information theory; locality-limited computation; entropy accounting; refinement dynamics. \\
\textbf{NON-CLAIMS:} No empirical claims; no statements about specific algorithms or models; no impossibility for global-invariant, oracle, or unbounded-memory systems.
\vspace{0.5em}
\hrule

\section{Background}
Local refinement procedures form a broad class of reasoning and computation systems.
Such systems update internal state through bounded-context observations and transformations,
without access to exact global invariants.
Classical results in finite model theory, descriptive complexity, and graph algorithms
establish strong limitations on what fixed-resource local procedures can distinguish.

Recent capacity-based analyses refine these locality limits by explicitly accounting for
the amount of information a system can retain across sequential updates.

\section{Motivation}
While single-pass or fixed-rule refinement systems exhibit clear capacity obstructions,
modern reasoning systems often employ adaptive strategies that compose multiple refinement
rules across interacting contexts.

This raises a natural question:
\emph{Do capacity-based obstructions persist when refinement itself is adaptive or multi-contextual?}

This manuscript isolates structural limits that remain invariant under such extensions.

\section{Intuition}
Adaptive refinement does not eliminate information constraints.
At each step, a system must still compress observations into a bounded internal transcript.
Even when multiple refinement channels interact, the total extractable information
remains bounded by aggregate capacity.

As a result, problems whose solution spaces carry linear or superlinear entropy
cannot be collapsed by adaptive refinement without exceeding system capacity.

\section{Refinement Systems with Interaction}
\begin{definition}[Interacting Refinement System]
An interacting refinement system consists of a finite collection of local refinement processes
whose internal states may influence one another through bounded-information channels,
without access to global invariants or unbounded persistent memory.
\end{definition}

Such systems generalize single-context refinement while preserving locality and capacity constraints.

\section{Aggregate Transcript Capacity}
\begin{definition}[Aggregate Transcript Capacity]
The aggregate transcript capacity $\mathrm{TC}_{\mathrm{agg}}$ of an interacting refinement system
is the supremum of mutual information it can retain, in the Shannon sense,
about the problem state across all internal channels and refinement steps.
\end{definition}

\section{Entropy Constraints Under Adaptivity}
Let $P$ be a problem with solution-space entropy $H(P)$.

\begin{lemma}[Capacity Additivity Bound]
For any interacting refinement system, the total extractable information across all channels
is bounded above by $\mathrm{TC}_{\mathrm{agg}}$.
\end{lemma}

\begin{lemma}[Adaptive Entropy Reduction Bound]
Adaptive composition of refinement rules does not increase expected per-step entropy reduction
beyond a constant factor depending only on $\mathrm{TC}_{\mathrm{agg}}$.
\end{lemma}

\begin{theorem}[Adaptive Entropy--Depth Lower Bound]
For any interacting refinement system with bounded aggregate transcript capacity,
solving a problem $P$ with $H(P)=\Theta(n)$ requires $\Omega(n)$ total refinement depth,
even under adaptive rule composition.
\end{theorem}

\section{Persistence of the Viotropic Regime}
\begin{definition}[Operational Mutual Information]
Let $X$ denote the joint internal state of an interacting refinement system
and $O$ its joint observation channel. Define
\[
I_{\mathrm{op}}(X;O) := H(O) - H(O \mid X).
\]
\end{definition}

\begin{definition}[Viotropic Information]
Information is \emph{viotropic} if $I_{\mathrm{op}}(X;O)=0$,
meaning it is operationally inaccessible despite internal presence.
\end{definition}

\begin{theorem}[Adaptive Viotropic Wall --- Conditional]
There exists $\kappa>0$ such that for any interacting refinement system,
\[
H_{\mathrm{extractable}} > \kappa \cdot \mathrm{TC}_{\mathrm{agg}}
\quad \Rightarrow \quad
I_{\mathrm{op}}(X;O)=0 .
\]
\end{theorem}

\section{Consequences}
Adaptive refinement does not circumvent structural capacity limits.

\begin{corollary}
No interacting refinement system with bounded aggregate transcript capacity
can reliably solve open-ended problems whose solution spaces exhibit unbounded entropy.
\end{corollary}

\section{Failure Modes}
The obstruction described here fails if:
\begin{itemize}
\item The system acquires exact global invariants.
\item Persistent memory or oracle access becomes unbounded.
\item Aggregate transcript capacity scales superlinearly with problem size.
\end{itemize}

\section{Relationship to Prior Work}
This result:
\begin{itemize}
\item Extends single-context capacity obstructions to adaptive systems.
\item Refines classical locality results into an explicit entropy framework.
\item Remains orthogonal to training-scale, optimization, or heuristic explanations.
\item Does not apply to kernel-verified proof assistants or symbolic systems with exact global checks.
\end{itemize}

\section*{References}
\begin{enumerate}
\item C.\ E.\ Shannon, \emph{A Mathematical Theory of Communication}, Bell System Technical Journal (1948).
\item L.\ Libkin, \emph{Elements of Finite Model Theory}, Springer (2004).
\item N.\ Immerman, \emph{Descriptive Complexity}, Springer (1999).
\item J.\ D.\ Hamkins, \emph{Can large language models do mathematics?}, blog (2023).
\end{enumerate}

\end{document}

